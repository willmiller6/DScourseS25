\documentclass{article}

% Language setting
% Replace `english' with e.g. `spanish' to change the document language
\usepackage[english]{babel}

% Set page size and margins
% Replace `letterpaper' with `a4paper' for UK/EU standard size
\usepackage[letterpaper,top=2cm,bottom=2cm,left=3cm,right=3cm,marginparwidth=1.75cm]{geometry}

% Useful packages
\usepackage{amsmath}
\usepackage{graphicx}
\usepackage[colorlinks=true, allcolors=blue]{hyperref}

\title{Problem Set 9}
\author{William Miller}

\begin{document}
\maketitle

\section{Problem Set Questions}

\subsection{Training Data Dimensions}
My housing\_test dataframe has 74 features while the original housing\_data dataframe had 14. That means we introduced 60 new features with our recipe

\subsection{LASSO Model}
My best $\lambda$ is 0.00139, with an in-sample RMSE of 1.96 and an out-of-sample RMSE of 1.95. I have also produced two diagnostic plots attached to this document.

\begin{figure}[htbp]
  \centering
  \includegraphics[width=0.8\textwidth]{LASSO CSV Lambda plot.png}
\end{figure}

\begin{figure}[htbp]
  \centering
  \includegraphics[width=0.8\textwidth]{LASSO prediction accuracy.png}
\end{figure}

\subsection{Ridge Model}
My best $\lambda$ was 0.0146, with an in-sample RMSE of 1.96 and an out-of-sample RMSE of 1.95. This was very similar to the LASSO model. Diagnostic plots for my ridge model are also available at the end of this document.

\begin{figure}[htbp]
  \centering
  \includegraphics[width=0.8\textwidth]{Ridge CV Lambda Plot.png}
\end{figure}

\begin{figure}[htbp]
  \centering
  \includegraphics[width=0.8\textwidth]{Ridge Prediction Accuracy.png}
\end{figure}

\subsection{OLS with k>n}
I don't believe OLS will produce a unique solution with more features than observations, which is why tools like LASSO and PCA are so useful for "big data" situations.

\section{Bias-Variance Tradeoff}
Because our in-sample and out-of-sample RMSE are small and quite similar for both models, I would say we are effectively balanced bias and variance.

\end{document}