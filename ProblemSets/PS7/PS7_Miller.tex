\documentclass{article}

% Language setting
% Replace `english' with e.g. `spanish' to change the document language
\usepackage[english]{babel}

% Set page size and margins
% Replace `letterpaper' with `a4paper' for UK/EU standard size
\usepackage[letterpaper,top=2cm,bottom=2cm,left=3cm,right=3cm,marginparwidth=1.75cm]{geometry}

% Useful packages
\usepackage{amsmath}
\usepackage{graphicx}
\usepackage[colorlinks=true, allcolors=blue]{hyperref}

\usepackage{booktabs}
\usepackage{siunitx}
\newcolumntype{d}{S[
    input-open-uncertainty=,
    input-close-uncertainty=,
    parse-numbers = false,
    table-align-text-pre=false,
    table-align-text-post=false
 ]}

\title{Problem Set 7}
\author{William Miller}

\begin{document}
\maketitle

\section{Question 1}

The first method (listwise deletion) is consistent with the MCAR assumption, meaning we assume any missingness in the logwage variable is fully attributable to random error or "noise." The second method, mean imputation, preserves the mean of the distribution but likely reduces the variance substantially, which is evident in the both the attenuated coefficient estimate and the reduced $R^2$ value.

The third and fourth methods are examples of imputation through regression, which is consistent with the MAR assumption. Model 3 (fitted dependent variable) is an example of single imputation whereas model 4 (multiple imputation) is an example of multiple imputation. The coefficient on model 4 is closer to the true coefficient (0.093).

These estimates are all relatively similar except for mean imputation. This indicates that mean imputation is the least effective method for these data.

\section{Question 2}

I will be examining the question "Does making the college football playoff increase the number of applications to a university in the following application cycle?" I will be using data I have collected from College Poll Archive, the College Football Playoff History website, IPEDS, the NCES, and FRED. I used an AR(1) model estimated Difference GMM. I have completed this project and find that making the associated press top 25 in the final week of polling is associated with a a 4-5\% increase in applications the following application cycle, whereas making the top 10, the playoffs, or winning the national championship are not assocated with any increase in applicants.

\begin{table}
\centering
\caption{Summary Table}
\centering
\begin{tabular}[t]{lrrrrrrr}
\toprule
  & Unique (\#) & Missing (\%) & Mean & SD & Min & Median & Max\\
\midrule
logwage & 670 & 25 & \num{1.6} & \num{0.4} & \num{0.0} & \num{1.7} & \num{2.3}\\
hgc & 16 & 0 & \num{13.1} & \num{2.5} & \num{0.0} & \num{12.0} & \num{18.0}\\
tenure & 259 & 0 & \num{6.0} & \num{5.5} & \num{0.0} & \num{3.8} & \num{25.9}\\
age & 13 & 0 & \num{39.2} & \num{3.1} & \num{34.0} & \num{39.0} & \num{46.0}\\
\bottomrule
\end{tabular}
\end{table}

\begin{table}
\centering
\caption{Regression Results}
\begin{tabular}[t]{lcccc}
\toprule
  & listwise deletion & Mean Imputation & Fitted Dependent Variable & Multiple Imputation\\
\midrule
(Intercept) & \num{0.657} & \num{0.849} & \num{0.657} & \num{0.646}\\
 & (\num{0.130}) & (\num{0.103}) & (\num{0.099}) & (\num{0.115})\\
hgc & \num{0.062} & \num{0.050} & \num{0.062} & \num{0.063}\\
 & (\num{0.005}) & (\num{0.004}) & (\num{0.004}) & \vphantom{1} (\num{0.005})\\
college\_grad & \num{-0.145} & \num{-0.168} & \num{-0.145} & \num{-0.139}\\
 & (\num{0.034}) & (\num{0.026}) & (\num{0.025}) & (\num{0.032})\\
tenure & \num{0.050} & \num{0.038} & \num{0.050} & \num{0.041}\\
 & (\num{0.005}) & (\num{0.004}) & (\num{0.004}) & (\num{0.005})\\
 tenure$^2$ & \num{-0.002} & \num{-0.001} & \num{-0.002} & \num{-0.001}\\
 & (\num{0.000}) & (\num{0.000}) & (\num{0.000}) & (\num{0.000})\\
age & \num{0.000} & \num{0.000} & \num{0.000} & \num{0.001}\\
 & (\num{0.003}) & (\num{0.002}) & (\num{0.002}) & (\num{0.003})\\
married\_dummy & \num{0.022} & \num{0.027} & \num{0.022} & \num{0.023}\\
 & (\num{0.018}) & (\num{0.014}) & (\num{0.013}) & (\num{0.017})\\
\midrule
Num.Obs. & \num{1669} & \num{2229} & \num{2229} & \num{2229}\\
Num.Imp. &  &  &  & \num{5}\\
R2 & \num{0.208} & \num{0.147} & \num{0.277} & \num{0.229}\\
R2 Adj. & \num{0.206} & \num{0.145} & \num{0.275} & \num{0.227}\\
AIC & \num{1179.9} & \num{1091.2} & \num{925.5} & \\
BIC & \num{1223.2} & \num{1136.8} & \num{971.1} & \\
Log.Lik. & \num{-581.936} & \num{-537.580} & \num{-454.737} & \\
F & \num{72.917} & \num{63.973} & \num{141.686} & \\
RMSE & \num{0.34} & \num{0.31} & \num{0.30} & \\
\bottomrule
\end{tabular}
\end{table}

\end{document}